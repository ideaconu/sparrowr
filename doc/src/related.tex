\normalfont\normalsize
\chapter{Related Work}
\label{chap:related}

% Truism
Energy harvesting has been studied for a long time and many tried to find a solution for their
shortcomings. Solar panels are frequently used in EHWSNs because they can generate a lot of power
but the power they generate is not consistent and predicting it was on of the challenges to be
overcome.

\section {Prediction generated Energy}

Because the solar energy is not constant, in order to predict the generated energy, a history of
past-days weather conditions or generated energy must be taken into account. The state-of-the-art
algorithm for this is Weather-Conditioned Moving Average (WCMA) \cite{piorno2009prediction} which
can keep a history of number of days of generated energy in order to predict the generated energy
in the next hour. The results of the algorithm are shown to be impressive, with an average error of
9.8\% in 45 days of testing.

Because the algorithm requires the generated power to be measured in order to have a precise
history of generated solar energy, this algorithm is not feasible in application where the node
can be very low power and a small solar panel of just 60mW or less can be used, compared to one
used in the paper, which could generate more than 300mW. A simpler solution for hardware as well as
software must be found for those applications.


\section {Using stored Energy}

The common approach to optimal packet scheduling is using a water-filling algorithm
\cite{yang2012optimal} \cite{huan2013utility}, where the time
is divided into slots and given a level of energy to be used in that slot. For better power
optimizations, this approach is modified into backward water-filling, directional water-filling,
generalized iterative water-filling \cite{want2015iterative} for offline (deterministic) scenarios. In order to simulate
online (stochastic) scenarios, Gaussian noise is added and the constant water level policy, energy
adaptive water-filling\cite{ozel2012optimal}, time-energy adaptive water-filling\cite{ozel2011transimision} can be used.
The problem with their approach, in our opinion, and even simpler ones like fuzzy power management
\cite{aoudia2016fuzzy} is that the complexity of the algorithm is rather high, at least O(N), a
constant feedback of consumed energy must be provided. The fact that the results of all the
algorithms were simulated and not implemented and tested on a real device, indicated that real
conditions, like capacitor leakage or temperature variations are not taken into consideration.
Also, every algorithm needs to know how much energy it is using performing different tasks. This is
not feasible in real world, in time leading to large errors.

What this mean is that the above algorithms will work best in high power scenarios, where leakage
currents are small enough to be considered inexistent. In a low power scenario, the results are
very optimal.

