\normalfont\normalsize
\chapter{Related Work}
\label{chap:related}

% Truism
Energy harvesting has been studied for a long time, and in spite of the advantage of free power,
there are some downsides. Solar panels are frequently used in EHWSNs because they can generate a
lot of energy but it is not consistent and predicting it was on of the challenges to be
overcome.

In this chapter we will present the current advances in solar powered EHWSN.

\section {Prediction generated Energy}

Because the solar energy is not constant, in order to predict the generated energy, a history of
past-days weather conditions or generated energy must be taken into account. The state-of-the-art
algorithm for this is Weather-Conditioned Moving Average (WCMA) \cite{piorno2009prediction}. IN
order to predict the generated energy in the next hour, it needs to keep a history of generated
energy for the past 6 days. The results of the algorithm are shown to be impressive, with an average error of
9.8\% in 45 days of testing.

Unfortunately, the algorithm requires the measurement of the generated energy in order to have an
exact history of generated solar energy. Because of this, the algorithm is not feasible in
applications where the node needs to be very low power due to the overhead of estimating the
generated power. Also the application could be powered by a small solar panel of just
60mW or less, much smaller compared to the one used in the paper, which could generate more than 300mW.
A simpler solution for hardware as well as software must be found for those applications.


\section {Using stored Energy}
The other downside of solar powered is the fact that in the night, rain or foggy conditions the
panel will generate a very small or next to nothing amount of energy. In order to keep to node
alive, transfer speed scheduling algorithms have been developed.

The common approach to optimal packet scheduling is using a water-filling algorithm
\cite{yang2012optimal} \cite{huan2013utility}, where the time
is divided into slots and given a level of energy to be used in that slot. For better power
optimizations, this approach is modified into backward water-filling, directional water-filling,
generalized iterative water-filling \cite{want2015iterative} for offline (deterministic) scenarios.
Real world applications are stochastic, so in order to simulate online scenarios, Gaussian noise is
added. The algorithms that can be used in this case are constant water level policy, energy
adaptive water-filling\cite{ozel2012optimal}, time-energy adaptive water-filling\cite{ozel2011transimision}.
Because these are complex algorithms, simpler ones have been attempted, such as fuzzy power
management\cite{aoudia2016fuzzy} , where a table with predetermined levels is used as main policy.
What this mean is that the above algorithms will work best in high power scenarios, where leakage
currents are small enough to be considered inexistent. Furthermore, the results of all the
algorithms were obtained using simulated data with programs such as
GreenCastalia\cite{benedetti2013greencastalia}. Because the algorithms were never tested on a
real device can mean that real conditions, like capacitor leakage or temperature variations are not taken into consideration.

All the above algorithms need to measure how much energy the node is using when performing
different tasks. This is not feasible in real conditions because precise current measurement can only be
performed at a high sample rate, which in turn consumes a large amount of energy. Considering the
presented problems, we developed a lightweight and efficient algorithm that needs to know only the
voltage of the capacitor.

