\chapter{Conclusions}


Wireless Sensor Networks are expanding more and more because they help make our lives easier by giving us information about our surroundings. However, the standard way of creating wireless sensor network islands is not always feasible.

The main goal of this thesis was to bring an alternative solution to the problem of how data can be obtained from Wireless Sensor Network islands. We have proven that a slightly modified low-cost drone offers a good range, being able to communicate with the nodes from a big distance. The communication range is greatly affected by the antenna gain and the objects that obstruct the line of sight from the drone to the node.


The solution, a truly mobile gateway is a viable one and can be implemented with a relatively low cost. The technical knowledge necessary to operate the system is not more complicated than knowing how to use a smartphone.


%Besides the practicality aspect of the solution, it can be used as a fun instrument too. A treasure hunt game can be played by placing the nodes with information regarding the location of the treasure at certain hidden spots and trying to find the treasure by following the clues provided by the drone. Other fun games that can be played is hiding the nodes in a certain zone. The players must find as many nodes as possible in that zone before the drone runs out of battery. The winner will be the player that found the largest number of nodes and has the best time.


\section{Future Work}

The proposed system can be improved in a number of ways. A feature that can be used in conventional wireless sensor networks is to determine the source of a communication failure. If the gateway detects that the network has a communication problem and not all of the previous nodes can be reached, a drone can use this information to search and find exactly which nodes are working properly and which nodes are not.

The Parrot AR.Drone 2.0 can perform autonomous flight with a GPS module, but only while it is still in the range of the Wi-Fi connection. This module can allow the drone to fly without the need to still be connected through Wi-Fi to a controlling device. A different autonomous flight mode can be implemented without a GPS module if the signal strength of the nodes is used to perform flight correction and determine the speed and direction of the drone.

