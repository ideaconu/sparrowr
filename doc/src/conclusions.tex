\chapter{Conclusions}

The main goal of this thesis was to present the complete architecture of a Energy Harvesting Wireless Sensor
Network. We have presented a brand new node designed to be new development platform that can be used to
bring new ideas to life. On of those ideas is dynamically varying the transmission speed in order to
keep alive a node until the capacitor can be recharged. We implemented an algorithm that we tested
on the node, something that our research found that is rarely done, and proved that it is able to
keep the node alive and use efficiently all the stored energy.


Because the algorithm needs to be able to read the voltage of the capacitor, the hardware
requirements are very small and it can be easily deployed on existing hardware with little to no
modification.

This research can be continued and for this we propose some directions presented in the next
section.

\section{Future Work}

Even though the algorithm is functional, there is always room for improvement. We would like our
work to be continued with an optimization of the algorithm for better time estimation. Taking
into consideration past weather in order to predict when we can start to generate power could help us to
better determine the new target time, so that in case of a very cloudy weather, we could adjust the
deadline and hopefully, keep the node alive until the sun will be able to recharge the
super-capacitor.

The evaluation of the algorithm revealed at least two more areas in which improvements are needed. First
is revealed by the test of the blue capacitor, in which at the end the send frequency increased.
This is not a negative aspect, as the deadline was reached, but a smaller difference between
consecutive changes in frequency might allow for a more constant energy discharge.

The second one is revealed by the 12 hours discharge test. The algorithm lowered the send frequency
down to 1 data per hour, but the time interval used to determine the new frequency reach 10 hours.
The solution would be that if the frequency is not changed in a determined time, that could be
affected by the type of capacitor done, the recalculations will have to be done for a better
estimation of the remaining duration.

Another point not covered in this thesis is multi-hop network. We tested the algorithm in a simple
single-hop scenario, but a multi-hop test is needed to know if the algorithm needs improvements for
this scenario.
