\normalfont\normalsize
\chapter{Introduction}


A big problem encountered when developing an application for  Wireless Sensor Networks is autonomy.
Big batteries can be used to power the nodes, but because some can be deployed in locations
difficult to reach, a simple task of changing the batteries becomes an impossible one.

The solution to this problem is powering the devices from alternative sources of energy, process
called energy harvesting. In the recent years, energy harvesting has become more and more used in the field of Wireless
Sensor Networks. There are plenty of alternative energy sources, such as solar cells, vibration
absorption generators, wind mills, thermoelectric generators and others, that can be used to power the
nodes or charge their batteries in order to become autonomous.

While the energy source problem has a solution, another problem appears in the form of finding a
method to store the generated energy. A battery could be used to store the generate energy, but
unfortunately, current technology allows to charge one for a few hundreds of cycles.
Considering that a cycle would be used per day, in maximum 2 years, the battery will lose most of
its original capacity. An alternative to the battery is using a super capacitor, which has a
lifetime of 10 years or 500.000 cycles, but are more expensive and store far less energy than regular
rechargeable batteries.

In order to alleviate the problem of the small amount of stored energy, a scheduling algorithm can
be used. The main goal of the algorithm is to dynamically vary the frequency with which the node
performs various tasks in order to be able to keep functioning between two sessions of generated
energy without losing power and to send as much data as possible.

In this thesis we will describe the architecture of an energy harvesting wireless sensor network,
shortly titled EHWSN. We will present a new node and development platform, the Sparrow R, designed
for low power and the problems we encountered when creating a EHWSN application. As the final part
of the architecture, we developed an efficient but lightweight algorithm for efficiently using the stored energy.


