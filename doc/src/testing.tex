\normalfont\normalsize
\chapter{Evaluation}

Our goal was to deliver a solution that could be deployed in an application as fast as possible, so
instead of running software simulations, we decided to implement and test the algorithm on the new
node Sparrow R. The algorithm is implemented in C, and compiled with gcc-arm-none-eabi-4.8.3-2014q1 on Arduino 1.6.4.

We selected 2 super-capacitors of 1F and 5V maximum rating. We fully charged the capacitor and then
let the algorithm decide how fast the data should be transmitted in order to reach the time
deadline.

The only input needed is the current voltage of the capacitor, read using a voltage divider that is
control by an n-mos transistor in order to reduce the power dissipated by the divider. The node
will run a task that simulates sensors readings and other processing through a delay of 100 ms. The
network is configure as single-hop, and the data sent through RF has the length of 45 bytes.

\begin{figure}[ht] \centering
\includegraphics[width=0.85\textwidth]{img/algtest.png}
\caption{Two runs of the algorithm}
\end{figure}

When beginning with a deadline of 4 hours, in the first run, the node manages to execute 1593 task and be
functional for a total time of 4 hours and 18 minutes. The second run, with a new deadline of 4
hours and 11 minutes, the node ran 1631 tasks for a total time of 4 hours and 35 minutes.

With a longer, more realistic deadline of 8 hours, the node managed to transmit 785 times for a
duration of 8 hours and 14 minutes. We ran into problems when a longer deadline of 12 hours was
tested, mainly because the self discharge of the capacitor combined with the idle current
consumption of the node is high enough to waste more 75\% of the energy. This had a significant
impact on the total number of executed tasks.


TODO - add the 12 hours test and talk more about it.

The 12 hour test meant that 1F capacitor is barely enough for 12 hours with our demo test and that
a bigger capacitor might be needed. Because we had two types of 1F capacitor, we were curios to
see if there is a difference between them. What we found was a bit of surprise, especially the
lowest voltage at which the node stopped working. The blue capacitor had a lower voltage of 1950mV
compared to the black of 2050mV. However, the black one managed to transmit data more times than
the blue one, even when the algorithm was tweaked to take into consideration the lower working
voltage of the blue one. The black managed to send 1642, while the blue one manage a 1254.

\begin{figure}[ht] \centering
\includegraphics[width=0.85\textwidth]{img/capacitors.jpg}
\caption{1F differen capacitors, Left Black - Right Blue}
\end{figure}

\begin{figure}[ht] \centering
\includegraphics[width=0.85\textwidth]{img/captest1.png}
\caption{1F different capacitors test, blue vs black}
\end{figure}


