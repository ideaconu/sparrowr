Nodes for wireless sensor networks have been created in the past ( voinescu), V3.2 is a successful
iteration based on Atmel Atmega128RFA1. Pared with the node, a dongle can be connected to a
computer using an usb port which allows the PC to act as a gateway for the network of nodes. Even
though the node is very low power, developing new applications on the node is very complicated. The
only way to program a node is using an ISP programmer. For debugging a JTAG must be used or a FTDI
for simple printf information display. Also, adding new hardware is very difficult, the expansions
    capabilities are limited which leads to using only the existing sensors mounted on the node.

The next iteration, the Sparrow V4 tried to fix the development environment by being Arduino
compatible. This removed the necessity of using an ISP for programming and having a simple printf
option via FTDI. The limited expansing capabilities are still an issue and unfortunately, the node
has a very high power consumption.


Tried this before using previous versions of sparrow. Sparrow E did not start on battery. Sparrow
V4.x has major power consumption problems , the idle power is 6.6mW. They are Arduino compatible but the platform is not fully tested and compatible with the processor.

The Arduino software not fully developed and hardware with designs flaws.
