\label{chap:results}


\subsection{Power Consumption}

Because the node was designed with low power in mind, we can obtain a very low deep sleep power
consumption of 5uW. A small solar panel together with a
small capacitor can be used as the main power supply.

For example, when a 1F capacitor is used with the voltage ranging from 3v3 to 2v1 the equivalent battery capacity would be

$$ \frac{F * (Vi - Vf)}{t} = \frac{1F * (3.3V - 2.1V)}{3600s} = 0.333mAh$$

The node is equipped with a DC/DC converter that can bring substantial power savings, depending on
the voltage of the power supply. A LDO wastes a lot off energy as the delta between the
voltages increases, were as DC/DC works best when the delta increases, or putting it simple, if an
LDO ouputs 1v8 and a chip consumes 5mA, the input current is almost the sam 5mA regardless of the
input voltage. So even though the chip consumes 9mW, the total power consumed is actually 25mW. If
in the same situation, a DC/DC with a 90\% efficiency is used, then the input current would be :
$$Iin = \frac{Iout * Vout}{Vin*Efficiency} = \frac{5mA * 1.8V*100}{5v*90}= 2mA$$


we can ignore the quiescent current because it is very small, around 360nA. The total power
consumption in this case is 10mW for an output power of 9mW compared with the LDO which consumes
25mW in order to output just 9mW, a 250\% difference between them.

In a real world situation, when a battery or a capacitor is used, the difference between them is
smaller. We will present two cases, in order to better understand the influence of higher
voltage supply.

The minimum voltage will be 2v1 for DC/DC and 1v9 for LDO. The current consumption of the chip is
5.5mA, and a capacitor of 1F.

Case 1: Capacitor charged to 3v3.

$$T_{LDO} = \frac{C * (V_i - V_f)}{I}=\frac{1F * (3.3V - 1.9V)}{5.5mA} = 254.45s $$

$$E_i = \frac{C*V_i^2}{2} = \frac{1f*(3.3V)^2}{2} =5.445J$$
$$E_f = \frac{C*V_f^2}{2} = \frac{1f*(2.1V)^2}{2} =2.205J$$
$$E = E_i - E_f = 3.24J$$
$$P = V*I*E_{fficiency }= \frac{1.8V * 5.5mA * 90}{100} = 11mW$$
$$T_{DC} = \frac{E}{P} = \frac{3.24J}{11mW} =294.54s $$

$$Advantage = \frac{T_{DC} - T_{LDO}}{T_{LDO}} = \frac{294.54s - 254.45s}{254.45s} = 15.7\%$$

Case 2: Capacitor charged to 5v.

$$T_{LDO} = \frac{C * (V_i - V_f)}{I}=\frac{1F * (5V - 1.9V)}{5.5mA} = 527.27s $$

$$E_i = \frac{C*V_i^2}{2} = \frac{1f*(5V)^2}{2} =12.5J$$
$$E = E_i - E_f = 10.295J$$
$$T_{DC} = \frac{E}{P} = \frac{3.24J}{11mW} =935.91s $$

$$Advantage = \frac{T_{DC} - T_{LDO}}{T_{LDO}} = \frac{935.91s - 527.27s}{527.27s} = 77.5\%$$

At 3.3V the advantage is not that big, but at 5V we nearly double the battery life. Furthermore, using a LIPO
battery which does not discharge as linearly as the capacitor will increase the
advantage of the DC/DC.

In real world testing, we charged a 1F capacitor up to 3v3 and let it discharge by transmitting data every second. Even though the software was not
optimized for low power, the node managed to run for 8 hours before the capacitor was completely
discharged which translates to an average power consumption for almost 112uW.


\subsection{Software}

The software is packed as a new board that can be installed on Arduino 1.6.x, latest iteration to date,
May 2016. It was tested using Windows 8.1, but it should be fully compatible with other operating systems as well.

The node has native USB which allows for code upload and also serial interface over CDC. Because no
extra components are needed, the same node can be configured to act as a gateway or
as a leaf.
